% ApJ Format
\documentclass[iop,revtex4,twocolumn,apj,numberedappendix,appendixfloats]{emulateapj}
\usepackage{apjfonts}
\usepackage{pdflscape}

% Hyperlinks & Bookmarks
\usepackage[pagebackref=false,letterpaper=true,colorlinks=true,citecolor=blue,linkcolor=blue,breaklinks=true,bookmarks=true]{hyperref}

% New commands
\newcommand{\eg}{e.g.,}
\newcommand{\ie}{i.e.,}
% Facilities
\newcommand{\chandra}{{\it Chandra}}
\newcommand{\hst}{{\it HST}}
\newcommand{\iras}{{\it IRAS}}
\newcommand{\rosat}{{\it ROSAT}}
\newcommand{\wise}{{\it WISE}}
\newcommand{\fir}{{FIRST}}
% Units
\newcommand{\kms}{{km s$^{-1}$}}
\newcommand{\msun}{$M_{\odot}$}
\newcommand{\lsun}{$L_{\odot}$}
\newcommand{\um}{$\mu$m}
\newcommand{\uJy}{$\mu$Jy}
\newcommand{\nd}{\nodata}
\newcommand{\sqdeg}{deg$^2$}
% emission/absorption lines
\newcommand{\Ha}{H$\alpha$}
\newcommand{\Hb}{H$\beta$}
\newcommand{\OII}{[O\,{\sc ii}]}
\newcommand{\OIIexpanded}{[O\,{\sc ii}]\,$\lambda$3727}
\newcommand{\SII}{[S\,{\sc ii}]}
\newcommand{\OIII}{[O\,{\sc iii}]}
\newcommand{\OIIIexpanded}{[O\,{\sc iii}]\,$\lambda$5007}
\newcommand{\NII}{[N\,{\sc ii}]}
\newcommand{\MgII}{Mg\,{\sc ii}}
% special 
\newcommand{\sersic}{S\'{e}rsic}
\newcommand{\nod}{\nodata}
\newcommand{\NOD}{\nodata}

\begin{document}

\title{SPFIT for MaNGA: Description of the Fitting Procedure}

\author{
Hai~Fu\altaffilmark{1} 
}
\altaffiltext{1}{Department of Physics \& Astronomy, University of Iowa, Iowa City, IA 52245}

\begin{abstract}
Here we describe the IDL data analysis package for MaNGA.
\end{abstract}

\keywords{galaxies: active --- galaxies: nuclei}

\section{MaNGA Observations and Data Reduction}

Mapping Nearby Galaxies at Apache Point Observatory (MaNGA; \citealt{Bundy15}) is an integral-field spectroscopic survey that is one of three core programs in the fourth-generation SDSS (see \citealt{Bundy15} for a survey overview). It utilizes 17  optical fiber-bundle integral field units (IFUs) to observe 10,000 nearby galaxies with the SDSS 2.5-meter telescope \citep{Gunn06}. For galaxy observations, there are five IFU sizes: the 19, 37, 61, 91, and 127-fiber bundles fill hexagonal areas with diameters of 12.5\arcsec, 17.5\arcsec, 22.5\arcsec, 27.5\arcsec, and 32.5\arcsec, respectively. Besides the science IFUs, each MaNGA cartridge also has 92 sky fibers and twelve 7-fiber mini-bundles for spectrophotometric calibration \citep{Yan16a}. The MaNGA galaxy sample is selected from the NASA-Sloan Atlas (NSA v1\_0\_1; Blanton M., \url{http://www.nsatlas.org}). It contains galaxies in a mass-dependent redshift range between $0.01 < z < 0.2$ (see \citet{Wake17} for the sample design). For each observation, these IFUs are plugged across a 3$^\circ$-diameter plate drilled for the target area. A three-point dithering pattern is employed to improve the uniformity and it increases the diameter of the FoV by $\sim$3.5\arcsec. The 2\arcsec\ fibers in combination with the BOSS spectrograph \citep{Smee13} yield spectral coverage 3,600$-$10,300 \AA\ at a typical resolution $R \sim 2000$. A number of dithered 15~min exposures are carried out for each plate to reach a S/N of 5 per \aa\ per fiber at an $r$-band surface brightness of 23 AB arcsec$^{-2}$. \citealt{Drory15} gives details of the IFU design, and \citet{Law15} presents simulations that guided the observing strategy of the survey. 

The raw data are reduced by a dedicated Data Reduction Pipeline \citep[DRP;][]{Law16}, which produces fully calibrated, stacked datacubes for each IFU, with a pixel size of 0\farcs5$\times$0\farcs5$\times$10$^{-4}$dex in (x,y,$\lambda$). The SDSS public data release 14 \citep[DR14;][]{Abolfathi17} includes 2,772 galaxy datacubes. The DR14 covers 2,720 unique galaxies, among which 317, 637, 639, 321, and 806 galaxies were observed using the 19, 37, 61, 91, and 127-fiber bundles, respectively. 41 galaxies were observed 2 or 3 times and for each of those we select the datacube that has the largest IFU size. When the IFU sizes are the same, we select the one with the highest SNR. 

\section{Spectral Fitting with SPFIT}

We developed our data analysis package {\sc SPFIT} in IDL. To prepare for spectral fitting, we first re-bin the DRP-produced cubes along the spatial dimensions to 1\arcsec$\times$1\arcsec\ pixels to remove much of the covariances among neighboring pixels, we then compute a total S/N map in the H$\alpha$ region (rest-frame 6525$-$6610~\AA) from the binned flux and inverse variance cubes, and finally we apply the Voronoi binning method \citep{Cappellari03} to the spatial dimensions to reach a target S/N of 30. We modified the Voronoi binning code to take the spatial covariance matrix instead of just the error vector. Ignoring the covariance among spaxels would significantly overestimate the binned S/N even for the 1\arcsec-pixel cubes. The covariance matrix is computed following exactly how the DRP samples individual fiber spectra to regularly sampled datacubes. Because of differential atmosphere refraction and the wavelength-dependent PSF, the covariance matrix is a function of wavelength. As an approximation, we use the covariance matrix computed for the central wavelength of the datacube. When computing the error vectors for the binned spectra for each Voronoi bin, we also account for their spatial covariances. Spectral covariance can be ignored because the DRP uses a method to maintain diagonal covariance when resampling and coadding fiber spectra (S. Bailey et al?). Foreground Galactic extinction is estimated from the standard dust map \citep{Schlegel98} and is corrected using the Galactic extinction law of \citet{Cardelli89} with $R_V = 3.1$. 

We model each spectrum as a number of common emission lines and a weighted sum of simple stellar populations (SSPs), convolved by the line-of-sight velocity distribution (LOSVD) and reddened in rest-frame by a Calzetti extinction law \citet{Calzetti00}. Both the LOSVD and the profile of the emission lines are parameterized as Gauss-Hermite series \citep{van-der-Marel93} to the fourth order. Both the SSP templates and the Gaussian emission-line templates are convolved to match the wavelength-dependent spectral resolution of each datacube. We used the SSP templates of MIUSCAT \citep{Vazdekis12}, which combines three empirical stellar libraries --- MILES \citep{Sanchez-Blazquez06}, CaT \citep{Cenarro01}, and Indo-U.S. \citep{Valdes04} --- to achieve a wide spectral range (3465$-$9469~\AA) with a uniform spectral resolution (FWHM = 2.5~\AA).  

To decrease the number of free parameters, we tie the kinematics of the forbidden lines and the recombination lines, with each group having their own independent kinematics. Because of its fast speed, we decided to use the Levenberg-Marquard nonlinear least-squares minimization algorithm \citep[][; chap. 15.5]{Press92} implemented in MPFIT \citep{Markwardt09}. But the success of the fitting (i.e., finding the global minimum) relies on a good set of initial guess. We thus separate our fitting into two parts (stellar continuum and emission lines) and uses the Penalized Pixel-Fitting method (pPXF; \citealt{Cappellari04}) to obtain the initial ``best-fit'' parameters. The pPXF method is very robust because it solves the weights of the templates with a linear algorithm \citep[BVLS;][]{Lawson74} {\it independently} from solving the Gauss-Hermite LOSVD with a nonlinear optimizer (MPFIT). We first run pPXF on the emission-line-masked spectrum to obtain the weights of the SSPs, their LOSVD, and the amount of extinction; we then run pPXF on the residual spectrum with the Gaussian emission-line templates to obtain their intrinsic line profile and the line fluxes. Finally, with this intermediate set of input parameters, which should be already quite close to the best-fit, we fit all of the parameters simultaneous and nonlinearly with MPFIT on the full observed spectrum. Following \citet{Cappellari04}, we penalize all of the $h_3$ and $h_4$ terms of the Gauss-Hermite series with an additional bias term in the residual calculation to stabilize the fit.

To deal with the small number of broad-line Seyfert galaxies in the sample, we include two additional components: (1) broad Balmer lines with independent line profiles from the narrow lines, and (2) a fourth order additive Legendre polynomial to approximate the AGN continuum. Once the best-fit kinematics and the shape of the AGN continuum are determined from the nuclear spectrum (which has the highest S/N), these parameters are fixed for the rest of the datacube because both components are supposed to emerge from spatially unresolved regions. The results are satisfactory for Seyferts in subclasses 1.5 through 2 \citep{Osterbrock81}. The code currently does not fit Seyfert 1s very well because additional components are needed (e.g., the blended Fe {\sc II} and Balmer continuum emission). We thus exclude from our analysis the two Seyfert 1s in the sample. %Some examples of our fitting results are given in Fig.~\ref{fig:fitting}.

MPFIT computes the formal errors of the free parameters by evaluating the covariance matrix of fitted parameters with the Jacobian matrix. This method tends to underestimate the error of the emission line fluxes because many of the parameters are tied together. So we perform Monte Carlo simulations to properly estimate the line flux errors. Because the total S/N of a line (i.e., $f$/d$f$) is mostly determined by two parameters: (1) the amplitude-to-noise ratio (A/N), and (2) the observed line width in pixels (instrumental $+$ intrinsic dispersion), after a suite of MC simulations covering a range of both parameters, we fit the function with a two-dimensional polynomial and evaluate any line flux errors based on the interpolated $f$/d$f$ for the best-fit A/N and line width. This method greatly saved the computational time.

%%%%%%%%%%%%%%%%%%%% REFERENCES %%%%%%%%%%%%%%%%%%
\bibliographystyle{apj}
\bibliography{exgal_ref}


\end{document}